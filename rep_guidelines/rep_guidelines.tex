
% "Станет проще"

\documentclass[a4paper,12pt]{article} % тип документа
\usepackage{cmap}
% report, book

%  Русский язык

\usepackage[T2B]{fontenc}            % кодировка
\usepackage[utf8]{inputenc}            % кодировка исходного текста
\usepackage{graphicx}
\usepackage[english,russian]{babel}    % локализация и переносы


%отступ
\usepackage[left=2cm,right=2cm,
    top=2cm,bottom=2cm,bindingoffset=0cm]{geometry}

% Математика
\usepackage{amsmath,amsfonts,amssymb,amsthm,mathtools}
\usepackage{csvsimple}
\usepackage{multirow}
\usepackage{wasysym}
\usepackage{subcaption}
\usepackage{verbatim}
%\usepackage{ae,aecompl} better q
\usepackage{float}
\usepackage{enumerate}
\usepackage[dvipsnames]{xcolor}
\usepackage{rotating}
\usepackage[hidelinks]{hyperref} % Hide links

%Заговолок
%\graphicspath{ {images/} }


\begin{titlepage}
\author{Соловьянов Михаил}
\title{Гайдлайн по составлению отчета к работе: "диод Шоттки"}
\date{\today}
\end{titlepage}



\begin{document} % начало документа
\maketitle


\section{Теоретический материал}

Убедительно советую изучить вот эти материалы:\\

\url{https://youtu.be/GtH8lAzQf2A}  - практическая электроника\\
\url{https://youtu.be/Y_OpjTYALLw} - теория часть 1\\
\url{https://youtu.be/yaPxtFLA3Ck} - теория часть 2\\

Можно изучить так же любые другие видео на русском или английском языке, и прочитать любые материалы. В том числе методичку.
\section{Ответы на вопросы}
Советую разделить области компетенции на три категории: \textbf{физика}, \textbf{техпроцессы}, \textbf{электроника}. Выбрать по одному (или больше) ответственному человеку который был бы готов все обьяснить.\\
Физика - это теория с предпосылками и объяснениями загадочного поведения материала в реальности.\\
Техпрооцесс - это все что связано с установками производства и измерения. Технологическая сторона вопроса.\\
Электроника - это практическое приложение полупроводникового прибора.\\

Вопросы можно найти тут: \textbf{\url{https://github.com/mikprin/shottky_lab/blob/master/questions/que.pdf}}

Отчеты других групп, и их данные доступны тут: \url{https://github.com/mikprin/shottky_lab}. Где номер папки соответствует номеру группы. При этом папка "data" содержит данные. А папка "rep" содержит отчет.

\section{Отчет}
В первую очередь не важен объем отчета! Смысл написания этих отчетов вообще, как и любых научных работ, это не перевод древесины, и не спонсирование производителей носителей цифровой информации, а попытка увековечить, и в разумной форме передать полученный вами опыт следующим поколениям, или в худшем случае другим специалистам. Конечно пусть этот отчет прочитаю только я и наши коллеги, но я прошу вас в меру серьезно подойти к этой методической задаче. У любого трактата есть мысль которую вы хотите донести. Она может быть не одна. У подобной исследовательской работы (реальной или учебной), в отличие от работы по философии, есть следующее задачи: рассказать, что же у вас вышло (это должно сразу быть видно тому, кому вообще лень читать вашу писанину). Рассказать как у вас это вышло,  (чтобы если человека все же очень впечатлило то, что вы кратко описали он мог бы перелистнуть страницу и попытаться понять в чем подвох ), ну и иногда можно написать почему это важно. Описывая то как у вас получился результат важно как раз описать, все что вам кажется важным для его повторения. Если вас не связывает коммерческая или военная тайна, то человек буквально должен прочувствовать как вы выполняете опыт. Это нужно, чтобы если вы совершили открытие и заявили о нем, было понятно как его повторить. В случае работы в закрытой фирме,  это облегчает жизнь вашим преемникам (примерно как комментарии в коде). Ну и наконец развернутые результаты должны быть представлены так, чтобы читатель мог прямо из них на бумаге или в PDF понять результат. Например, он же не может увеличить в PDF интересующий его участок, или в уме прикинуть наклон аппроксимирующей прямой. Делайте так, чтобы можно было увидеть результат. Пишите так, как хотели бы прочитать сами если бы пришли бы на лабораторную работу.\\
В целом вы вольны использовать любой стиль который вам кажется уместным, важно именно донести информацию. Когда моего коллегу теорфизика просят изложить статью с ограничением в 1 график, то это недели ручной работы. Каждый график у него уникален. У вас ограничений нет, но важно чтобы было не скучно читать. Теперь к плану.
В целом мне кажется удачным вот такой план работы:

\begin{enumerate}
  \item КРАТКИЙ результат. Кто вы и что сделали. Сразу и громко.
  \item Мотивация. Почему-то что вы делали вообще заслуживает вашего времени.
  \item Какая-то физика, если вы углубились в нее. И тут важно выделить именно ту ее часть которая важна. Например, если вы нашли обьяснение какому то феномену, или неидеальности в задаче. Не нужно переписывать учебник (это нарушение авторских прав). Лучше учебника вы все равно не напишите. Но например объяснить разницу в результатах математикой тут можно.
  \item Все про постановку эксперимента. Что и как делали. На чем делали. Все чтобы можно было повторить.
  \item Исследование результата. Графики это не просто доказательство вашего прибывания на паре, а скорее доказательство какой-то информации.
  \item Ну и итоговый вывод. Развернуто повторяет первую часть.
\end{enumerate}
Этот порядок можно менять, чтобы передать лучше смысл, я лишь показал как оно может быть.

\section{Требования}
\begin{itemize}
  \item Чтобы весь график ВАХ диода имел смысл его нужно строить в логарифмированном масштабе, но не ограничеваясь им.
  \item Отдельно интересно сравнить токи утечки, и прямые напряжения диодов. Поскольку это их определяющие характеристики.
  \item Интересно посмотреть на все коэффициенты эквивалентного сопротивления диодов.
  \item Четко обозначайте модели диодов.
  \item Любую информацию о покупных диодах можно найти в datasheet соответствующего диода. Нужно уметь читать эти документации.
  \item Хочется чтобы в вашем отчете был кусок, который похож на документацию для вашего устройства.
\end{itemize}


\end{document}
